% おわりに

本研究で開発したシステムは,
2021年度9月より実施される学部横断型PBLのアプリ開発を
行う際に使用される予定である.
開発したシステムから,メンバの背景と行為やメンバ間の交渉に着目し,
アプリの機能やUIがどのように決定されていくか分析する予定である.
加えてアプリ開発支援ソフトウェアがアプリ開発の過程にどのように影響を及ぼすかまで含めて分析を行う.

また,本研究において開発した可視化システムについて,
ノードの数が増えた際の可視性の改善が必要だと考えられる.
改善方法については,
論文の共著関係やSNSを対象としたネットワーク可視化の先行研究をもとに検討する予定である.

また,今後の展望として,
目標とするメンバの関係性のあり方に合わせて,
システムからプロジェクトメンバへの協同作業の提案を行うことも考えられる.
これにより,リソースへのアクセスができていないメンバを正統的周辺参加の状態にするきっかけを作ることができると考えられる.
