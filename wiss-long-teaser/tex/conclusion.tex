
本研究で開発したシステムは,
2021年度9月より実施される学部横断型PBLのアプリ開発を
行う際に使用される予定である.
開発したシステムから,メンバの背景と行為やメンバ間の交渉に着目し,
アプリの機能やUIがどのように決定されていくか分析する予定である.
加えてアプリ開発支援ソフトウェアがアプリ開発の過程にどのように影響を及ぼすかまで含めて分析を行う.

また,本研究において開発した可視化システムについて,
いくつかの改善点があげられる.
まず,メンバがタスクを行った回数によって,
ノードの半径を変化させている点についてである,
半径が変化することによって,
ノード間の可視性に影響を与えてしまうと考えられるので,
ノードの彩度や明度を使った表現に変更することが考えられる.
つぎに,現在のグラフ構造による可視化手法についてである.
この問題点としては,
メンバの数が増えた際にネットワークが複雑になってしまい,
いわゆる毛玉問題の様な現象が起きてしまうことが考えられる.
よって,メンバが増えてネットワークが複雑になった場合の,
ノードのレイアウト変更や,インタラクションの追加が考えられる.
これらの点についても,
学部横断型PBLのアプリ開発の際に改善点を精査し,修正する予定である.

今後の展望として,
目標とするメンバの関係性のあり方に合わせて,
システムからプロジェクトメンバへの協同作業の提案を行うことも考えられる.
これにより,リソースへのアクセスができていないメンバを正統的周辺参加の状態にするきっかけを作ることができると考えられる.
