
チームメンバーをマネジメントする観点から,
可視化手法を用いた研究は多く存在する.
本章では関連研究として,アプリケーション開発における
異なる実践共同体の参加の過程を可視化するシステム研究について,
Trello APIを使用したアジャイル開発におけるチームメンバーの
技術的なタスクに関連する感情の情報を可視化するシステムと,
オンライン上の実践共同体をD3.jsを用いて可視化したシステム,
それぞれ本研究との差異を述べ,新規性を示す.

\subsection{Trello APIを用いたアジャイル支援システム}
AmrらのEmotimonitor\cite{Emotimonitor}はチームメンバが
自分の感情をTrelloのカードに
絵文字で表現することを可能にした上で,
これらを分析し,統計データを提供するものである.
上記の機能により,アジャイル開発におけるタスクに関する
メンバの感情状態を認識できることにより,
マネージャーがチームメンバの感情の状態を評価し、
チームメンバが自分の感情を振り返ることを
支援することが目的としてある.

AmrらはTrelloのカードに絵文字という感情表現の機能を
拡張することによりあるチームメンバの内面を可視化し
プロジェクトの管理を支援しているのに対して,
本研究ではTrello APIを用いて,プロジェクトに参加する
チームメンバ間の関係性から
プロジェクトの実践を支援することを目的としている.   
\subsection{オンライン上の実践共同体の可視化}
YUBOら\cite{D3jsOfCop}は急速に変化する
UX(user experience)デザインのスキルが,
オンライン上のコミュニティでどのように学習されているかを
実践共同体の概念から分析を行っている.
分析したデータを用いて,D3.jsを使用してオンライン上の実践共同体の
メンバ同士のインタラクションをネットワークとして可視化している.

YUBOらの研究では実践共同体の概念を用いて,
UXデザインという共通した目的を持つ実践共同体のメンバのやり取りを
D3.jsで可視化しているのに対し,
本研究では共通の目的を共有している実践共同体のみを分析するのではなく,
異なる目的をもつ複数の実践共同体のやり取りを含めて
分析してD3.jsで可視化することに特徴がある.